%Modified template for a CV. Not pretty code, in fact not great code, but it works.

%%%%%

\documentclass{resume} % Use the custom resume.cls style
 \usepackage{bibentry}
%\usepackage{times}
\usepackage[T1]{fontenc}
 \usepackage[hidelinks]{hyperref}
\usepackage{url}
\urlstyle{tt}
\usepackage{etaremune}
\usepackage[left=1 in,top=1in,right=1 in,bottom=1in]{geometry} % Document margins
\newcommand{\tab}[1]{\hspace{.2667\textwidth}\rlap{#1}} 
\newcommand{\itab}[1]{\hspace{0em}\rlap{#1}}
\newcommand{\updateinfo}[1][\today]{\par\vfill{\textit{Last updated on #1}}}
\name{Dylan Schlichting} % Your name

\pagenumbering{Roman}
\pagestyle{plain}
\begin{document}
\begin{minipage}[ht]{0.35\textwidth}
  \href{http://www.tamu.edu/}{Texas A\&M University} \\
  \href{http://ocean.tamu.edu/}{Department of Oceanography} \\
  618 Eller O$\&$M Building  \\
  College Station, TX 77843-3146
\end{minipage}
\begin{minipage}[ht]{0.6\textwidth}
  \href{mailto:dylan.schlichting@tamu.edu}{dylan.schlichting@tamu.edu} \\
  (413) 262-4393\\
  \url{https://dylanschlichting.github.io/}
  \updateinfo
\end{minipage}
\vspace{-65 pt}
\bibliographystyle{apalike}
\nobibliography{Manuscripts}
%----------------------------------------------------------------------------------------
%	EDUCATION SECTION
%----------------------------------------------------------------------------------------
\vspace{60pt}
\begin{rSection}{Education}

Ph.D. Oceanography, Texas A$\&$M 
University \hfill {Jan 2020 - Dec 2024 (Expected)
}
\\
\textit{Advisors}: Drs. Robert Hetland $\&$ Henry Potter
\\
\\
B.S. Civil Engineering, University of Maine \hfill {Aug 2016 - Dec 2019}
\\ 
{\textit{Minor}:} Mathematics
\end{rSection}

%	WORK EXPERIENCE SECTION
\begin{rSection}{Research Experience}
\begin{rSubsection}{Graduate Research Assistant}{Jan 2020 - Present}{Texas A$\&$M University: Dept. Oceanography}{} 
\end{rSubsection}    

\begin{rSubsection}{Student Research Assistant}{May 2017 - Dec 2019}{UMaine: Dept. Civil Engineering}{} 
\end{rSubsection}
%------------------------------------------------

\begin{rSubsection}{Engineering Research Assistant}{Aug 2018 - May 2019}{UMaine: School of Marine Sciences}{}
\end{rSubsection}

\begin{rSubsection}{Research Experience for Undergraduates}{May 2018 - Aug 2018}{Texas A$\&$M University: Dept. Oceanography}{} 
\end{rSubsection} 
\end{rSection} 
\vspace{-3pt}

\begin{rSection}{Research Interests}
\vspace{-10pt}
\item High resolution coastal ocean modeling, spurious mixing, submesoscale processes and dynamics, estuarine exchange flow and mixing

\end{rSection}

\begin{rSection}{Publications}
\vspace{-3pt}

\begin{etaremune}
\item \textbf{Schlichting, D.}, Qu, L., Kobshi, D., and Hetland, R. Quantification of physical and numerical mixing in a coastal ocean model using salinity variance budgets. \textit{Journal of Advances in Modeling Earth Systems}. In revision.
\item Qu, L., Hetland, R., and \textbf{Schlichting, D.} Mixing pathways in simple box models (2022). \textit{Journal of Physical Oceanography}, 52(11), 2761-2772. \url{https://doi.org/10.1175/JPO-D-22-0074.1}.
\item Spicer, P., \textbf{Schlichting, D.}, Huguenard, K., Roche, A., and Rickard, L. (2021). Sensing Storm Surge: A framework for establishing a citizen scientist monitored water level network. \textit{Ocean and Coastal Management}, 211, 105802. \url{https://doi.org/10.1016/j.ocecoaman.2021.105802}.
\end{etaremune}
\end{rSection}

\begin{rSection}{Invited talks/lectures}
\vspace{-3pt}

\begin{etaremune}
    \item \textbf{Schlichting, D.} (2022). An introduction to numerical mixing in a coastal ocean model of the Texas-Louisiana continental shelf. Submesoscales under near-resonant inertial shear experiment (SUNRISE) meeting. December 11. Bend, OR.
    \item \textbf{Schlichting, D.}, Qu, L., Hetland, R., and Kobashi, D. (2022). Quantification of physical and numerical mixing using tracer variance dissipation in a coastal ocean model. Pacific Northwest National Laboratory coastal modeling group. July 11.
\end{etaremune}
\end{rSection}

\begin{rSection}{Academic Presentations} \itemsep -3pt
\begin{etaremune}
    \item \textbf{Schlichting, D.}, Qu, L., Hetland, R., and Kobashi, D. (2022). Quantification of physical and numerical mixing using tracer variance dissipation in a coastal ocean model. Gordon Research Seminar/Conference on Ocean Mixing, June 4-10. Poster.
    \item Hetland, R., Qu, L., and \textbf{Schlichting, D.} (2022). Tracer variance mixing in simple box models. Ocean Sciences Meeting. February 24 - March 4. Talk.
    \item \textbf{Schlichting, D.}, Qu, L., Hetland, R., and Kobashi, D. (2022). Using salinity variance budgets to quantify numerical mixing in a coastal ocean model. Ocean Sciences Meeting. February 24 - March 4. Talk. 
    \item \textbf{Schlichting, D.}, Hetland, R., Qu, L., and Kobashi, D. (2021). Using tracer variance budgets to quantify numerical mixing offline in a coastal ocean model. Warnem\"{u}nde Turbulence Days Meeting. December 6-9. Talk. 
    \item \textbf{Schlichting, D.}, Lieberthal, B., and Huguenard, K. (2019). An assessment into vegetation farms as a solution to coastal erosion in southern Maine. Northeast Aquaculture Conference, Boston MA. January 9-11. Poster.
    \item \textbf{Schlichting, D.} and Hetland, R. (2018). Using salinity variance and total exchange flow to analyze salinity structure in an unsteady estuary. Physics of Estuaries and Coastal Seas Conference, Galveston TX. October 14-18. Poster.
    \item \textbf{Schlichting, D.} and Hetland, R. (2018). Mechanisms controlling salinity structure structure in a broad, shallow, unsteady estuary. Sustainable Ecological Aquaculture Network Undergraduate Research Symposium, Walpole ME. August 7. Poster.
    \item \textbf{Schlichting, D.} and Hetland, R. (2018). Salinity structure in Copano Bay. Texas A$\&$M University Observing the Ocean REU Student Symposium, College Station, TX. August 2. Talk.
    \item \textbf{Schlichting, D.}, Lieberthal, B., and Huguenard, K. (2017). Vegetation farms as a solution to coastal erosion for Saco, Maine. Sustainable Ecological Aquaculture Network Undergraduate Research Symposium, Walpole ME. August 16. Poster.
\end{etaremune}
\end{rSection}

\begin{rSection}{Additional conferences/workshops} \itemsep -3pt
\begin{etaremune}
    \item Texas Center for Climate Studies High Resolution Earth System Modelling Workshop (2023). College Station, TX, Jan. 23-25. 
    \item Scientific Computing with Python Conference (2021). Jul. 12-18. \textit{Virtual}. 
    \item Scientific Computing with Python Conference (2020). Jul. 6-12.  \textit{Virtual}.
    \item Coastal and Estuarine Research Federation Conference (2017). Providence, RI, Nov. 5-9.
\end{etaremune}
\end{rSection}

\begin{rSection}{Service and teaching}
NSF PROGRESS Mentor \hfill Fall 2022 \\
Judge: Environmental Geosciences (GEOS 405, TAMU) \hfill Spring 2022 \\
Tutor: Computers in Civil Engineering (CIE 115, UMaine) \hfill Spring 2019 \\
\end{rSection}
\vspace{-10pt}
%----------------------------------------------
\begin{rSection}{Honors and Awards} \itemsep -3pt {}
\vspace{-7pt}
\item Louis and Elizabeth Scherck Scholarship \hfill 2020-Present
\item NSF S-STEM Scholar \hfill Jan 2020 - Aug 2021 
\item Oceanography Graduate Council mini-grant recipient (X3) \hfill 2021
\item Frank Sleeper - Sawyer Scholarship \hfill 2017 - 2019
\item Best capstone project \hfill 2019
\item Chi Epsilon Member: Civ. Eng. Honors Society \hfill 2019 
\item NSF Research Experience for Undergraduates Scholar \hfill May 2018 - Aug 2018
\item Alpha Tau Omega Memorial Scholarship \hfill 2018
\end{rSection}

\begin{rSection}{SKILLS}
\begin{itemize}
%\begin{rSubsection}{Computing and Programming}{}{}{} 
    \item Proficient in Python, Matlab, and Markdown
    \item Proficient in \LaTeX
    \item Basic experience with FORTRAN, C, C\#
    \item Website design with GitHub Pages, HTML, and Ruby
    \item Basic experience with Linux administration
    \item Experience designing and analyzing Regional Ocean Modeling System (ROMS) simulations
\end{itemize}
%\item Basic knowledge of Gener
%\end{rSubsection}
%\begin{rSubsection}{Civil Engineering}{}{}{} 
%\item Experience with Autocad, Revit, HEC-RAS, and Microsoft Project
%\end{rSubsection}
\end{rSection}

\begin{rSection}{Professional Societies} \itemsep -3pt {}
\vspace{-7pt}
\item Association for the Sciences of Limnology and Oceanography
\item The Oceanography Society 
\item American Society of Civil Engineers
\end{rSection}


%----------------------------------------------------------------------------------------

\end{document}
