\documentclass{resume} % Use the custom resume.cls style
 \usepackage{bibentry}
%\usepackage{times}
\usepackage[T1]{fontenc}
 \usepackage[hidelinks]{hyperref}
\usepackage{url}
\urlstyle{tt}
\usepackage[left=1 in,top=1in,right=1 in,bottom=1in]{geometry} % Document margins
\newcommand{\tab}[1]{\hspace{.2667\textwidth}\rlap{#1}} 
\newcommand{\itab}[1]{\hspace{0em}\rlap{#1}}
\newcommand{\updateinfo}[1][\today]{\par\vfill{\textit{Last updated on #1}}}
\name{Dylan Schlichting} % Your name

\pagenumbering{Roman}
\pagestyle{plain}
\begin{document}
\begin{minipage}[ht]{0.35\textwidth}
  \href{http://www.tamu.edu/}{Texas A\&M University} \\
  \href{http://ocean.tamu.edu/}{Department of Oceanography} \\
  618 Eller O$\&$M Building  \\
  College Station, TX 77843-3146
\end{minipage}
\begin{minipage}[ht]{0.6\textwidth}
  \href{mailto:dylan.schlichting@tamu.edu}{dylan.schlichting@tamu.edu} \\
  (413) 262-4393\\
  \url{https://dylanschlichting.github.io/}
  \updateinfo
\end{minipage}
\vspace{-65 pt}
\bibliographystyle{apalike}
\nobibliography{Manuscripts}
%----------------------------------------------------------------------------------------
%	EDUCATION SECTION
%----------------------------------------------------------------------------------------
\vspace{60pt}
\begin{rSection}{Education}

Ph.D. Oceanography, Texas A$\&$M 
University \hfill {Jan 2020 - Dec 2024 (Expected)
}
\\
\textit{Adviser}: Drs. Robert Hetland $\&$ Henry Potter
\\
\\
B.S. Civil Engineering, University of Maine \hfill {Aug 2016 - Dec 2019}
\\ 
{\textit{Minor}:} Mathematics
\\
\\
Relevant Coursework: Physical Oceanography, Numerical Methods, Ocean-Atmosphere Dynamics, Partial Differential Equations 
\end{rSection}

%	WORK EXPERIENCE SECTION
\begin{rSection}{Research Experience}
\begin{rSubsection}{Graduate Research Assistant}{Jan 2020 - Present}{Texas A$\&$M University: Dept. Oceanography}{Adviser: Dr. Robert Hetland} 
\end{rSubsection}    

\begin{rSubsection}{Student Research Assistant}{May 2017 - Dec 2019}{UMaine: Dept. Civil Engineering}{Adviser: Dr. Kimberly Huguenard} 
\end{rSubsection}
%------------------------------------------------

\begin{rSubsection}{Engineering Research Assistant}{Aug 2018 - May 2019}{UMaine: School of Marine Sciences}{Adviser: Dr. Neal Pettigrew}
\end{rSubsection}

\begin{rSubsection}{Research Experience for Undergraduates}{May 2018 - Aug 2018}{Texas A$\&$M University: Dept. Oceanography}{Adviser: Dr. Robert Hetland} 
\end{rSubsection} 
\end{rSection} 
\vspace{-3pt}

\begin{rSection}{Research Interests}
\vspace{-10pt}
\item Coastal ocean modeling, submesoscale processes, estuarine physics, ocean mixing

\end{rSection}

\begin{rSection}{Publications}
\vspace{-10pt}
\item Spicer, P., \textbf{Schlichting, D.}, Huguenard, K., Roche, A., and Rickard, L. Sensing Storm Surge: A framework for establishing a citizen scientist monitored water level network. \textit{Ocean and Coastal Management}. In press.
\end{rSection}

\begin{rSection}{Presentations and Conferences} \itemsep -3pt  
\begin{enumerate}
    \item \textbf{Schlichting, D.}, Lieberthal, B., and Huguenard, K. (2019). An assessment into vegetation farms as a solution to coastal erosion in southern Maine. Northeast Aquaculture Conference, Boston MA. January 9-11. Poster.
    \item \textbf{Schlichting, D.} and Hetland, R. (2018). Using salinity variance and total exchange flow to ana-lyze salinity structure in an unsteady estuary. Physics of Estuaries and Coastal Seas Conference, Galveston TX. October 14-18. Poster.
    \item \textbf{Schlichting, D.} and Hetland, R. (2018). Mechanisms controlling salinity structure structure in a broad, shallow, unsteady estuary. Sustainable Ecological Aquaculture Network Undergraduate Research Symposium, Walpole ME. August 7. Poster.
    \item \textbf{Schlichting, D.} and Hetland, R. (2018). Salinity structure in Copano Bay. Texas A$\&$M University Observing the Ocean REU Student Symposium, College Station, TX. August 2. Talk.
    \item \textbf{Schlichting, D.}, Lieberthal, B., and Huguenard, K. (2017). Vegetation farms as a solution to coastal erosion for Saco, Maine. Sustainable Ecological Aquaculture Network Undergraduate Research Symposium, Walpole ME. August 16. Poster.
\end{enumerate}
\end{rSection}

\begin{rSection}{Teaching}
Tutor: Computers in Civil Engineering (CIE 115, UMaine) \hfill Spring 2019 \\
\end{rSection}
\vspace{-10pt}
%----------------------------------------------
\begin{rSection}{Honors and Awards} \itemsep -3pt {}
\vspace{-7pt}
\item NSF S-STEM Scholar \hfill Jan 2020 - Aug 2021 
\item Oceanography Graduate Council mini-grant recipient  \hfill Summer 2021
\item Louis and Elizabeth Scherck Scholarship \hfill 2020, 2021  
\item Frank Sleeper - Sawyer Scholarship \hfill 2017 - 2019
\item Best capstone project \hfill 2019
\item Chi Epsilon Member: Civ. Eng. Honors Society \hfill 2019 
\item NSF REU Scholar \hfill May 2018 - Aug 2018
\item Alpha Tau Omega Memorial Scholarship \hfill 2018
\end{rSection}

\begin{rSection}{SKILLS}
\begin{rSubsection}{Computing and Programming}{}{}{} 
\item Proficient in Python - used for graduate studies and research
\item Proficient in Matlab - used for undergraduate studies and research
\item Basic experience with FORTRAN and C (see Ocean Modeling)
\item Proficient in \LaTeX
\item Basic experience with Linux administration
\end{rSubsection}
\begin{rSubsection}{Ocean Modeling}{}{}{} 
\item ROMS - idealized and realistic simulations of shelf circulation
\item Basic experience with SUNTANS
\end{rSubsection}
\begin{rSubsection}{Civil Engineering}{}{}{} 
\item Experience with Autocad, Revit, HEC-RAS, and Microsoft Project
\end{rSubsection}
\end{rSection}

\begin{rSection}{Professional Societies} \itemsep -3pt {}
\vspace{-7pt}
\item Association for the Sciences of Limnology and Oceanography
\item The Oceanography Society 
\item American Society of Civil Engineers
\end{rSection}


%----------------------------------------------------------------------------------------

\end{document}
