\documentclass{resume} % Use the custom resume.cls style
 \usepackage{bibentry}
 \usepackage{times}
 \usepackage{hyperref}
\usepackage{url}
%\urlstyle{same}
\usepackage[left=1 in,top=1in,right=1 in,bottom=1in]{geometry} % Document margins
\newcommand{\tab}[1]{\hspace{.2667\textwidth}\rlap{#1}} 
\newcommand{\itab}[1]{\hspace{0em}\rlap{#1}}
\name{Dylan Schlichting} % Your name
\address{University of Maine, Boardman Hall, Orono, ME 04469} % Your address
\address{+1-413-262-4393 \\  \href{mailto:dylan.schlichting@maine.edu}{dylan.schlichting@maine.edu}
}  % Your phone number and email
\pagenumbering{Roman}
\pagestyle{plain}
\begin{document}
\vspace{-85 pt}
\bibliographystyle{apalike}
\nobibliography{Manuscripts}
%----------------------------------------------------------------------------------------
%	EDUCATION SECTION
%----------------------------------------------------------------------------------------

\begin{rSection}{Education}

{\bf B.S.} Civil Engineering \hfill {August 2016 - December 2019}
\\ 
{\bf Minor:} Mathematics
\\
GPA 3.451 / 4.0 
\\
Relevant Coursework: Physical Oceanography (Graduate), Coastal Engineering (Graduate), Numerical Methods in Engineering (Graduate), Partial Differential Equations

\end{rSection}

%	WORK EXPERIENCE SECTION
\begin{rSection}{Research Experience}
\begin{rSubsection}{Student Research Assistant}{May 2017 - Present}{UMaine: Dept. Civil Engineering}{Adviser: Dr. Kimberly Huguenard} 
\item Used analytical modeling to study the interaction of flow and suspended kelp farms  
\item Analyzed the environmental impacts of coastal armoring structures on pocket beaches in Southern Maine
\item Constructed, deployed, and managed an oceanographic mooring system for the Sensing Storm Surge Citizen Science Project
\item Calculated and analyzed storm surge in three estuaries in mid-coast Maine
\item Designed and managed laboratory experiments to study hydrodynamic and turbulent characteristics of flow through floating oyster cages
\end{rSubsection}
%------------------------------------------------

\begin{rSubsection}{Engineering Research Assistant}{August 2018 - May 2019}{UMaine: School of Marine Sciences}{Adviser: Dr. Neal Pettigrew}
\item Characterized inertial oscillations in the Gulf of Maine using observational data
\item Used spectral analysis to study ocean circulation in the Gulf of Maine
\end{rSubsection}

\begin{rSubsection}{Research Experience for Undergraduates}{May 2018 - August 2018}{Texas A$\&$M University: Dept. Oceanography}{Adviser: Dr. Robert Hetland} 
\item Analyzed Regional Ocean Modeling Systems outputs in Python to study salinity structure in Copano Bay, TX
\item Analyzed estuarine time scales and total exchange flow in Copano Bay to understand poor performance of existing numerical models
\item Cruise: R/V Pelican (3 days). Cocodrie, LA, to Flower Garden Banks National Marine Sanctuary 
\end{rSubsection} 
\end{rSection} 
\vspace{-3pt}
\begin{rSection}{Teaching}
Tutor: Computers in Civil Engineering \hfill Spring 2019 \\
\end{rSection}
\vspace{-10pt}
%	Presentations
\begin{rSection} {Manuscripts in Preparation} \itemsep -3pt
\begin{enumerate}
    \item \bibentry{Stormsurge}
    \item \bibentry{TechnicalReport}
    \item \bibentry{OysterExperiment}
\end{enumerate}    
\end{rSection}

\begin{rSection}{Other Technical Writing} \itemsep -3pt 
\begin{enumerate}
    \item  Schlichting, D., Daubert, C., Dyer, A., Earl-Johnson, D., and Zachau, C. (2019). Maine Ocean School final design report. Department of Civil and Environmental Engineering senior capstone project. 228 pp.
    \item  Schlichting, D., Daubert, C., Dyer, A., Earl-Johnson, D., and Zachau, C. (2018). Maine Ocean School project proposal. Department of Civil and Environmental Engineering senior capstone project. 58 pp.
\end{enumerate}
\end{rSection}

\begin{rSection}{Presentations and Conferences} \itemsep -3pt  
\begin{enumerate}
    \item \bibentry{Northeast2019}
    \item Schlichting, D. and Hetland, R. (2018). Using salinity variance and total exchange flow to ana-lyze salinity structure in an unsteady estuary. Physics of Estuaries and Coastal Seas Conference, Galveston TX. October 14-18. Poster.
    \item Schlichting, D. and Hetland, R. (2018). Mechanisms controlling salinity structure structure in a broad, shallow, unsteady estuary. Sustainable Ecological Aquaculture Network Undergraduate Research Symposium, Walpole ME. August 7. Poster.
    \item Schlichting, D. and Hetland, R. (2018). Salinity structure in Copano Bay. Texas A$\&$M University Observing the Ocean REU Student Symposium, College Station, TX. August 2. Talk.
    \item \bibentry{SEANET2017}
\end{enumerate}
\end{rSection}
%----------------------------------------------
\begin{rSection}{Honors and Awards} \itemsep -3pt {}
\vspace{-7pt}
\item Frank Sleeper - Sawyer Scholarship \hfill August 2017 - Present
\item Best capstone project \hfill May 2019
\item Chi Epsilon Member: Civ. Eng. Honors Society \hfill May 2019 
\item Dean's List \hfill Fall 2016, Spring 2019
\item Alpha Tau Omega Memorial Scholarship \hfill December 2018
\end{rSection}


\begin{rSection}{SKILLS}
\begin{tabular}{ @{} >{\bfseries}l @{\hspace{6ex}} l }
Programming & MATLAB, Python, R, and C$\#$  \\
Modeling & Regional Ocean Modeling Systems and FVCOM\\
Oceanographic Equipment &  Experience with ADCPs, ADVs, CTDs, and PTVs \\
Civil Engineering & Autocad, Revit, and HEC-RAS \\ 
Other & \LaTeX, Wolfram Mathematica, Microsoft word, excel, and project 
\end{tabular}
\end{rSection}
\newpage
\begin{rSection}{Professional Societies} \itemsep -3pt {}
\vspace{-7pt}
\item Association for the Sciences of Limnology and Oceanography
\item The Oceanography Society 
\item American Society of Civil Engineers
\end{rSection}
%----------------------------------------------------------------------------------------
\begin{rSection}{Extra-Curricular Activities} \itemsep -1pt {}   
\vspace{-7pt}
\item Alpha Tau Omega \hfill Fall 2017 - Present
\item Engineers Without Borders \hfill Fall 2016
\end{rSection}
\end{document}
