%Modified template for a CV. Not pretty code, in fact not great code, but it works.

%%%%%

\documentclass{resume} % Use the custom resume.cls style
 \usepackage{bibentry}
%\usepackage{times}
\usepackage[T1]{fontenc}
 \usepackage[hidelinks]{hyperref}
\usepackage{url}
\urlstyle{tt}
\usepackage{etaremune}
\usepackage[left=1 in,top=1in,right=1 in,bottom=1in]{geometry} % Document margins
\newcommand{\tab}[1]{\hspace{.2667\textwidth}\rlap{#1}} 
\newcommand{\itab}[1]{\hspace{0em}\rlap{#1}}
\newcommand{\updateinfo}[1][\today]{\par\vfill{\textit{Last updated on #1}}}
\name{Dylan Schlichting} % Your name

\pagenumbering{Roman}
\pagestyle{plain}
\begin{document}
\begin{minipage}[ht]{0.35\textwidth}
  \href{https://www.lanl.gov/org/ddste/aldsc/theoretical/fluid-dynamics-solid-mechanics/index.php}{Los Alamos National Laboratory} \\
  Fluid Dynamics \& Solid mechanics group,
  TA-03 200  \\
  Los Alamos, NM 87545
\end{minipage}
\begin{minipage}[ht]{0.6\textwidth}
  \href{mailto:dylan.schlichting@tamu.edu}{dylan.schlichting@tamu.edu} \\
  (413) 262-4393\\
  \url{https://dylanschlichting.github.io/}
  \updateinfo
\end{minipage}
\vspace{-65 pt}
\bibliographystyle{apalike}
\nobibliography{Manuscripts}
%----------------------------------------------------------------------------------------
%	EDUCATION SECTION
%----------------------------------------------------------------------------------------
\vspace{60pt}
\begin{rSection}{Education}

Ph.D. Oceanography, Texas A$\&$M 
University \hfill {Jan 2020 - Aug 2024 (Expected)
}
\\
Committee: Robert Hetland (co-chair), Henry Potter (chair), Spencer Jones, Scott Socolofsky
\\
Dissertation: Numerical mixing in submesoscale baroclinic instabilities over sloping bathymetry 
\\
\\
B.S. Civil Engineering, University of Maine \hfill {Aug 2016 - Dec 2019}
\\ 
{\textit{Minor}:} Mathematics.  \\ \textit{Honors: cum laude}
\end{rSection}

\begin{rSection}{Research Interests}
\vspace{-10pt}
\item Spurious/numerical mixing, coastal ocean modeling, submesoscale processes and dynamics, estuarine exchange flow and mixing.

\end{rSection}

%	WORK EXPERIENCE SECTION
\begin{rSection}{Research Experience}
\begin{rSubsection}{DOE SCGSR Fellow}{Dec 2023 - Present}{Los Alamos National Laboratory}{} 
\item Assessing the role of coastal processes in the Model for Prediction Across Scales - Ocean (MPAS-O) over the Texas-Louisiana (TXLA) shelf
\item Learning grid generation and simulation development with unstructured ocean models
\end{rSubsection}    
\begin{rSubsection}{Graduate Research Assistant}{Jan 2020 - Present}{Texas A$\&$M University: Dept. Oceanography}{} 
\item Characterized numerical mixing in a high-resolution ocean model (COAWST/ROMS) of the TXLA shelf as part of the Submesoscales Under Near-Resonant Inertial Shear Experiment (SUNRISE, \url{https://sunrise-nsf.github.io/})
\item Developed idealized ROMS simulations of submesoscale baroclinic instabilities for a coastal shelf
\item Contributed to the development of modern, python-based post-processing packages for C-grid ocean model output (xroms, xarray, xgcm)
\end{rSubsection}    

\begin{rSubsection}{Student Research Assistant}{May 2017 - Dec 2019}{UMaine: Dept. Civil Engineering}{} 
\item Used analytical modeling to study the interaction of flow and suspended kelp farms.  
\item Analyzed the environmental impacts of coastal armoring structures on pocket beaches in Southern Maine using empirical orthagonal functions.
\item Participated in the construction, deployment, and management of an oceanographic mooring system for the Sensing Storm Surge Citizen Science Project. (\url{http://sensingstormsurge.acg.maine.edu/}).
\end{rSubsection}
%------------------------------------------------

\begin{rSubsection}{Engineering Research Assistant}{Aug 2018 - May 2019}{UMaine: School of Marine Sciences}{}
\item Characterized inertial oscillations in the Gulf of Maine using observational current data.
\end{rSubsection}

\begin{rSubsection}{Research Experience for Undergraduates}{May 2018 - Aug 2018}{Texas A$\&$M University: Dept. Oceanography}{} 
\item Characterized salinity structure in Copano Bay, TX using ROMS model output.
\item Cruise: R/V Pelican (3 days). Cocodrie, LA, to Flower Garden Banks National Marine Sanctuary in the northern Gulf of Mexico.
\end{rSubsection} 
\end{rSection} 
\vspace{-3pt}

\begin{rSection}{Publications}
\vspace{-3pt}

\begin{etaremune}
\item \textbf{Schlichting, D.}, Qu, L., Kobashi, D., \& Hetland, R. (2023). Quantification of physical and numerical mixing in a coastal ocean model using salinity variance budgets. \textit{Journal of Advances in Modeling Earth Systems},
15, e2022MS003380. \url{https://doi.org/10.1029/2022MS003380}.
\item Qu, L., Hetland, R., \& \textbf{Schlichting, D.} Mixing pathways in simple box models (2022). \textit{Journal of Physical Oceanography}, 52(11), 2761-2772. \url{https://doi.org/10.1175/JPO-D-22-0074.1}.
\item Spicer, P., \textbf{Schlichting, D.}, Huguenard, K., Roche, A., \& Rickard, L. (2021). Sensing Storm Surge: A framework for establishing a citizen scientist monitored water level network. \textit{Ocean and Coastal Management}, 211, 105802. \url{https://doi.org/10.1016/j.ocecoaman.2021.105802}.
\end{etaremune}
\end{rSection}

\begin{rSection}{Manuscripts in preparation}
\vspace{-3pt}
\begin{etaremune}
\item \textbf{Schlichting, D.}, \& Hetland, R. Numerical mixing suppresses submesoscale baroclinic instabilities over sloping bathymetry.
\item Wei Hsu, F., Shearman, R. Kipp , \textbf{Schlichting, D.}, Kobashi, D., \& Hetland, R. $S_2$ Atmospheric Tide Driven Superinertial Oscillation on the Texas-Louisiana Shelf. Intent to submit to \textit{Journal of Physical Oceanography}.
\end{etaremune}
\end{rSection}

\begin{rSection}{Invited presentations}
\vspace{-3pt}

\begin{etaremune}
    \item \textbf{Schlichting, D.} (2022). An introduction to numerical mixing in a coastal ocean model of the Texas-Louisiana continental shelf. SUNRISE student cruise meeting. Dec 11. Bend, OR. \textit{Talk}.
    \item \textbf{Schlichting, D.}, Qu, L., Hetland, R., \& Kobashi, D. (2022). Quantification of physical and numerical mixing using tracer variance dissipation in a coastal ocean model. Pacific Northwest National Laboratory coastal modeling group. Jul 11. \textit{Talk, virtual}.
\end{etaremune}
\end{rSection}

\begin{rSection}{Academic Presentations / Conferences} \itemsep -3pt
\begin{etaremune}
    \item \textbf{Schlichting, D.}, \& Hetland, R. (2023). Numerical mixing in idealized simulations of baroclinic instabilities over a shelf. Gordon Research Seminar/Conference on coastal ocean dynamics, Jun 17-23. \textit{Poster}.
    \item Texas Center for Climate Studies High Resolution Earth System Modelling Workshop (2023). College Station, TX, Jan 23-25. \textit{Attended}.
    \item \textbf{Schlichting, D.}, Qu, L., Hetland, R., \& Kobashi, D. (2022). Quantification of physical and numerical mixing using tracer variance dissipation in a coastal ocean model. Gordon Research Seminar/Conference on ocean mixing, Jun 4-10. \textit{Poster}.
    \item Hetland, R., Qu, L., \& \textbf{Schlichting, D.} (2022). Tracer variance mixing in simple box models. Ocean Sciences Meeting. Feb 24 - Mar 4. \textit{Talk}.
    \item \textbf{Schlichting, D.}, Qu, L., Hetland, R., \& Kobashi, D. (2022). Using salinity variance budgets to quantify numerical mixing in a coastal ocean model. Ocean Sciences Meeting. Feb 24 - Mar 4. Talk. 
    \item \textbf{Schlichting, D.}, Hetland, R., Qu, L., \& Kobashi, D. (2021). Using tracer variance budgets to quantify numerical mixing offline in a coastal ocean model. Warnem\"{u}nde Turbulence Days Meeting. Dec 6-9. \textit{Talk}. 
    \item Scientific Computing with Python Conference (2021). Jul 12-18. \textit{Attended, virtual}. 
    \item Scientific Computing with Python Conference (2020). Jul 6-12.  \textit{Attended, virtual}.
    \item \textbf{Schlichting, D.}, Lieberthal, B., \& Huguenard, K. (2019). An assessment into vegetation farms as a solution to coastal erosion in southern Maine. Northeast Aquaculture Conference, Boston MA. Jan 9-11. \textit{Poster}.
    \item \textbf{Schlichting, D.} \& Hetland, R. (2018). Using salinity variance and total exchange flow to analyze salinity structure in an unsteady estuary. Physics of Estuaries and Coastal Seas Conference, Galveston TX. October 14-18. Poster.
    \item \textbf{Schlichting, D.} \& Hetland, R. (2018). Mechanisms controlling salinity structure structure in a broad, shallow, unsteady estuary. Sustainable Ecological Aquaculture Network Undergraduate Research Symposium, Walpole ME. August 7. Poster.
    \item \textbf{Schlichting, D.} \& Hetland, R. (2018). Salinity structure in Copano Bay. Texas A$\&$M University Observing the Ocean REU Student Symposium, College Station, TX. August 2. \textit{Talk}.
    \item \textbf{Schlichting, D.}, Lieberthal, B., \& Huguenard, K. (2017). Vegetation farms as a solution to coastal erosion for Saco, Maine. Sustainable Ecological Aquaculture Network Undergraduate Research Symposium, Walpole ME. August 16. \textit{Poster}.
    \item Coastal and Estuarine Research Federation Conference (2017). Providence, RI, Nov 5-9. \textit{Attended}.
\end{etaremune}
\end{rSection}

\begin{rSection}{Service \& Mentoring}
Mentor: Kaila Uyeda (Postbac Researcher) \hfill Aug. 2023 - Dec. 2023 \\
% Reviewer: \textit{Journal of Advances in Modeling Earth Systems} \hfill Aug. 2023 - Present \\
Reviewer: \textit{Journal of Geophysical Research: Oceans} ($n=1$) \hfill Aug. 2023 - Present \\
Judge: Student Research Week (multiple sessions) \hfill Spring 2023 \\
NSF PROGRESS Mentor - Milly Hencey \hfill Fall 2022 \\
Judge: Environmental Geosciences (GEOS 405, TAMU) \hfill Spring 2022 \\
Tutor: Computers in Civil Engineering (CIE 115, UMaine) \hfill Spring 2019 \\
\end{rSection}
\vspace{-10pt}
%----------------------------------------------
\begin{rSection}{Honors and Awards} \itemsep -3pt {}
\vspace{-7pt}
\item DOE Office of Science and Graduate Student Research Program (SCGSR) \hfill Dec. 2023-Present
\item Louis and Elizabeth Scherck Scholarship (X4) \hfill 2020-Present
\item NSF S-STEM Scholar (X2) \hfill Jan 2020 - Aug 2021 
\item Oceanography Graduate Council mini-grant recipient (X3) \hfill 2021
\item Frank Sleeper - Sawyer Scholarship \hfill 2017 - 2019
\item Best civil engineering capstone project \hfill 2019
\item Chi Epsilon Member: Civ. Eng. Honors Society \hfill 2019 
\item NSF Research Experience for Undergraduates Scholar \hfill May 2018 - Aug 2018
\item Alpha Tau Omega Memorial Scholarship \hfill 2018
\end{rSection}

\begin{rSection}{SKILLS}
\begin{itemize}
%\begin{rSubsection}{Computing and Programming}{}{}{} 
    \item Ocean modeling: Experience with ROMS/COAWST \& MPAS-O, 
    \item Programming \& Related: Python, Matlab, \LaTeX, Markdown, Bash, FORTRAN, Linux administration, high performance computing
    \item Basic website design with GitHub Pages, HTML, Ruby, and CSS
    \item Ocean observations: HOBO water level and conductivity sensors, basic experience with ADCPs and ADVs
    \item Open science: GitHub/git, Zenodo, GLOBUS
    \item Civil engineering: Concrete mooring design, basic experience with HEC-RAS, AutoCad, and Revit
\end{itemize}
\end{rSection}

\begin{rSection}{Professional Societies} \itemsep -3pt {}
\begin{itemize}
    \item American Geophysical Union
    \item Association for the Sciences of Limnology and Oceanography
    \item The Oceanography Society 
\end{itemize}
\end{rSection}


%----------------------------------------------------------------------------------------

\end{document}
